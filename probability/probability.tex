\documentclass[12pt]{article}
\usepackage[margin=1in]{geometry}
\usepackage{epsfig}
\usepackage{amsfonts}
\usepackage[mathcal]{euscript} 
\usepackage[tbtags]{amsmath}
\usepackage{amssymb}
\usepackage{mathrsfs}
\usepackage{import}
\usepackage{hyperref}
\usepackage{minted}
\usepackage{enumerate}

\renewcommand{\P}{\mathbb{P}}

\newcommand{\example}[3]{%
    \vspace{.15in}

    \noindent \textbf{Example~#1}\nopagebreak

    \noindent #2

    \noindent \textbf{Solution}: #3
}


\newcommand{\cond}[1]{\textbf{\textcolor{blue}{#1}}}
\title{Basic Probability Notes}
\author{Ramesh Sridharan\thanks{Contact: \mbox{rameshvs@csail.mit.edu}}}
\date{}

\begin{document}
    \maketitle
    These notes give a review of basic probability, focusing on a few important
    concepts.  For a more thorough treatment, see any introductory probability
    book; I recommend \emph{Introduction to Probability} by Bertsekas \& Tsitsiklis.
    \section*{Probability review}
    Here's our setup: we have \textsl{experiments} which generate
    \textsl{outcomes}.  The \textsl{sample space} is the set of all outcomes of
    an experiment.  \textsl{Events} are subsets of the sample space (which can
    possibly be empty). A \textsl{probability distribution} assigns
    probabilities to events, and must satisfy the three axioms of
    nonnegativity, additivity, and normalization.

    \textsl{Conditioning} on an event means we have observed that event
    occuring.  This gives a new probability distribution; it can be thought of
    as restricting the universe of possibilities to those outcomes which
    satisfy the condition.

    Two events $A$ and $B$ are defined to be \textsl{independent} if $\P(A|B) =
    \P(A)$, or equivalently if $\P(A \cap B) = \P(A)\P(B)$. Intuitively, two
    events are independent if knowing one does not change the probability of
    the other one occurring.

    \textsl{Bayes' rule} states that
    \begin{align*}
        \P(A|B) &= \frac{\overbrace{\P(B|A)}^{\text{``likelihood''}}\overbrace{\P(A)}^{\text{``prior''}}}{\P(B)}.
    \end{align*}
    Typically, $B$ is an observed event and $A$ is an event representing the
    hidden state of the world (so $P(A|B)$ is our belief about the hidden
    state of the world given our observations). Our problem-specific model will
    provide likelihood and prior distributions $P(B|A)$ and $P(A)$
    respectively. Computing the denominator often requires the use of total
    probability (or, in the case of random variables, marginalization), and can
    sometimes be computationally intractable.

    The law of \textsl{total probability} states that given events $A_1,
    \ldots, A_n$ that are disjoint (i.e., they don't overlap) and exhaustive
    (i.e., they cover the entire sample space when combined), we can write
    \begin{align*}
        \P(B) &= \sum_{i=1}^n \P(B|A_i) \P(A_i)
    \end{align*}
    This is useful when $\P(B)$ is difficult to compute, but $\P(B|A_i)$ is easier.

    Also recall that the notation $A^c$ refers to the complement of an event.

    \subsection*{Basic combinatorics}

    Suppose we wish to form a 4-letter string using the letters of the
    alphabet. Suppose we require that the letters must be different (i.e.\ we
    sample them \textit{without replacement}).  Then, there are 26
    possibilities for the first letter, and for each of those, there are 25
    possibilities for the second letter, and so on.  So, the number of possible
    strings is given by $26 \cdot 25 \cdot 24 \cdot 23$.

    In general, a length-$k$ \textbf{permutation} of $n$ elements is an ordered
    collection of $k$ of the $n$ elements. There are $\frac{n!}{(n-k)!}$
    different ways to construct such a permutation. In this example, $n=26$
    letters of the alphabet and $k=4$ letters, so the formula gives us $26!/22!
    = 26 \cdot 25 \cdot 24 \cdot 23$.

    When we are not concerned with order, we can simply take the number of
    permutations and divide by the number of ways to reorder the $k$ elements,
    $k!$. This gives $\frac{n!}{k!(n-k)!}$.

    In our earlier example, if we didn't care about order, we would divide by
    the number of ways to rearrange the 4 letters (which is $4! = 4 \cdot 3 \cdot
    2$). Our final answer would then be $\frac{26!}{22! \cdot 4!} = \frac{26
    \cdot 25 \cdot 24 \cdot 23}{4 \cdot 3 \cdot 2}$.

    \clearpage
    \section*{Problems}

    \example{1}{%
        A family has two children. For each piece of information below,
        determine the probability that both children are boys given that piece
        of information. For this problem, assume that every child is
        equally likely to be a boy or a girl, and the genders of the two
        children are independent.
        \begin{enumerate}[(a)]
            \item No information (i.e.\ compute the unconditional probability that both children are boys).
            \item The younger child is a boy.
            \item At least one child is a boy.
            \item At least one child is a boy born on a Tuesday. Assume each
            child is equally likely to be born on any day of the week.
        \end{enumerate}
    }{%
    This problem was posed by Gary Foshee at a gathering of puzzle enthusiasts.
    You can find lots of discussion about it online~\footnote{See
    \href{http://www.newscientist.com/article/dn18950-magic-numbers-a-meeting-of-mathemagical-tricksters.html}{the
    New Scientist article} or \href{http://wmbriggs.com/blog/?p=2553}{a useful
    blog post}}.
    \begin{enumerate}[(a)]
        \item This is simply the probability of two independent events, each of which occurs with
            probability $1/2$. Therefore, the probability is $\frac{1}{4}$.
        \item Carefully examining the sample space, we see that there are 4 possible outcomes:
            \begin{table}[h!]
                \centering
                \begin{tabular}{|c|c|}
                    \hline
                    First child & Second child \\ \hline
                    \cond{Boy} & \cond{Boy} \\ \cond{Boy} & \cond{Girl} \\ Girl & Boy \\ Girl & Girl \\ \hline
                \end{tabular}
            \end{table}

            Using computations similar to those in (a), we can see that all 4
            possibilities are equally likely. If we observe that the younger
            child is a boy, we can eliminate the third and fourth
            possibilities. One out of the two remaining possibilities satisfies
            the event we are looking for (``both children are boys''). So, the
            probability is $\frac{1}{2}$.
        \item Once again, we can write out the entire sample space:
            \begin{table}[h!]
                \centering
                \begin{tabular}{|c|c|}
                    \hline
                    First child & Second child \\ \hline
                    \cond{Boy} & \cond{Boy} \\ \cond{Boy} & \cond{Girl} \\ \cond{Girl} & \cond{Boy} \\ Girl & Girl \\ \hline
                \end{tabular}
            \end{table}

            Here, our conditioning event, ``at least one child is a boy,'' only
            eliminates one possibility, leaving 3 equally likely candidates.
            Once again, only the first out of the three satisfies the event
            we're looking for. So, the probability is $\frac{1}{3}$.
        \item In order to write out the full sample space for this problem,
            we'd need 196 entries: there are 14 possibilities for the first child,
            and for each of those, 14 possibilities for the second.  So, we'll
            solve a simpler problem: suppose babies can only be born on Tuesdays
            and Wednesdays.  Using B, G, T, and W to abbreviate boy, girl, Tuesday
            and Wednesday respectively, the sample space is:
            \begin{table}[h!]
                \centering
                \begin{tabular}{|c|c|}
                    \hline
                    First child & Second child \\ \hline
                    \cond{B,T} & \cond{B,T} \\
                    \cond{B,T} & \cond{B,W} \\
                    \cond{B,T} & \cond{G,T} \\
                    \cond{B,T} & \cond{G,W} \\
                    \hline
                    \cond{B,W} & \cond{B,T} \\
                    B,W & B,W \\
                    B,W & G,T \\
                    B,W & G,W \\
                    \hline
                    \cond{G,T} & \cond{B,T} \\
                    G,T & B,W \\
                    G,T & G,T \\
                    G,T & G,W \\
                    \hline
                    \cond{G,W} & \cond{B,T} \\
                    G,W & B,W \\
                    G,W & G,T \\
                    G,W & G,W \\
                    \hline
                \end{tabular}
            \end{table}

            The 7 bolded entries satisfy the event we're conditioning on: that
            one of the children is a boy born on a Tuesday. Out of these, only
            3 (the first two and the fifth) correspond to two-boy families.
            Therefore, the probability is $\frac{3}{7}$.

            While the sample space for the full problem is too large to write
            out easily, we can extend this reasoning to obtain the final answer
            $\frac{13}{27}$. I encourage you to try arriving at this answer for
            yourself!

    \end{enumerate}
    In case you don't believe the answer (as I didn't the first time I
    heard this puzzle), one thing you can always do is to simulate it.
    The following python code (also available from the same place you
    found these notes) simulates this problem:

    \inputminted[linenos,frame=lines,framesep=1mm]{python}{probability.py}
    One way to view these problems is on an ``information spectrum:'' In each
    situation, we are given some information about the family, and asked for
    the probability of having two boys. This probability was smallest in (a),
    where we were given no information. As we are given more and
    more information, the probability of this event increases: part (c)
    provides a small amount of information, part (d) provides a little more,
    and part (b) provides even more still. Our smaller problem in part (d)
    doesn't provide as much information as the full problem: we only eliminate
    half the possible days the child could have been born on. However, the full
    problem eliminates $6/7$ of the days by telling us that the boy was born on
    a Tuesday.}

    \example{2}{%
        You have an urgent assignment due in only a few minutes. You know
        it's in one of your drawers, but you're not sure which one.  The
        probability that the assignment is in drawer $k$ is $D_k$.  If drawer
        $k$ has the assignment and you search there, you have probability
        $p_k$ of finding it.

        Suppose you search drawer $i$ and do not find the assignment. Like any
        good student, you decide to spend your little remaining time computing
        probabilities:
        \begin{enumerate}[(a)]
            \item Find the probability that the paper is in drawer $j$, where $j \neq i$.
            \item Find the probability that the paper is in drawer $i$.
        \end{enumerate}
    }{%
        Let $A_k$ be the event that the assignment is in drawer $k$, and $B_k$ be
        the event that you find the assignment in drawer $k$.
        \begin{enumerate}[(a)]
            \item We'll express the desired probability as $\P(A_j|B_i^c)$. Since
                this quantity is difficult to reason about directly, we'll use
                Bayes' rule:
                \begin{align*}
                    %\overbrace{\P(A_j|B_i^c)}{\text{Prob. assignment in drawer $j$ given it wasn't found in drawer $i$}} &= \frac{\P(B_i^c | A_j) \P(A_j)}{\P(B_i)}
                    \P(A_j|B_i^c) &= \frac{\P(B_i^c | A_j) \P(A_j)}{\P(B_i^c)}
                \end{align*}
                The first probability in the fraction, $\P(B_i^c | A_j)$, expresses
                the probability of not finding the assignment in drawer $i$ given
                that it's in a different drawer $j$. Since it's impossible to find
                the paper in a drawer it isn't in, this is just 1.

                The second probability, $\P(A_j)$, is given to us in the problem
                statement as $D_j$.

                The third probability, $\P(B_i^c) = 1-\P(B_i)$, is difficult to
                reason about directly. But, if we knew whether or not the paper was
                in the drawer, it would become easier. So, we'll use total
                probability:
                \begin{align*}
                    \P(B_i) &= \overbrace{\P(B_i|A_i)}^{\text{prob.\ of finding in $i$ if in $i$}}\underbrace{\P(A_i)}_{\text{prob.\ of being in $i$}} + \overbrace{\P(B_i|A_i^c)}^{\text{prob.\ of finding in $i$ if not in $i$}}\underbrace{\P(A_i^c)}_{\text{prob.\ of not being in $i$}}\\
                    &= p_i D_i + 0 (1-D_i)
                \end{align*}
                Putting these terms together, we find that
                \begin{align*}
                    \P(A_j|B_i^c) &= \frac{D_j}{1-p_iD_i}
                \end{align*}
            \item Similarly, we'll use Bayes' rule:
                \begin{align*}
                    \P(A_i | B_i^c) &= \frac{\P(B_i^c | A_i) \P(A_i)}{\P(B_i)} = \frac{(1-p_i) D_i}{1 - p_i D_i}
                \end{align*}
        \end{enumerate}
    }
    \example{3}{%
        Suppose you have $n$ ball-filled urns, numbered $1$ through $n$. Urn
        $i$ has exactly $i-1$ green balls and $n-i$ red balls.
        \begin{enumerate}[(a)]
            \item You choose an urn uniformly at random, and draw a ball
                without looking at it. What is the probability that the ball you drew is green?
            \item You then look at the ball and see that it is green. You draw
                another ball from the same urn. What is the probability that
                the second ball is green?
        \end{enumerate}
    }{%
    \begin{enumerate}[(a)]
        \item Let $N_\text{red}$ be the total number of red balls, and $N_\text{green}$ be the
            total number of green balls. 
            \begin{align*}
                N_\text{red} &= \sum_{i=1}^n (i-1) = \frac{1}{2}n(n+1) - n = n^2-n,  \\
                N_\text{green} &= \sum_{i=1}^n (n-i) = n^2 - \frac{1}{2}n(n+1) = n^2-n, 
            \end{align*}
            where the summations come from the formula $\sum_{i=1}^n i = \frac{1}{2}n(n+1)$.

            Since the urns all have an equal number of balls, randomly choosing
            an urn and then randomly choosing a ball is equivalent to randomly
            choosing a ball from all the balls (convince yourself that this is
            true!). So, the probability is $1/2$. This answer should also make
            intuitive sense: since red and green are completely symmetric here,
            the answer should be the same for red as it is for green: the only
            way for that to be true is for both to be $1/2$.
        \item This question would be much easier if we knew which urn the balls
            came from. So, we'll use total probability to condition on that information. Let
            $G_k$ be the event that the $k$th ball drawn is green, and $U_i$ be the
            event that they were drawn from the $i$th urn. Then the $U_i$s are
            disjoint, exhaustive events:
            \begin{align*}
                \P(G_2 | G_1) &= \frac{\P(G_2 \cap G_1)}{\P(G_1)} \\
                    \text{(total probability)} 
                        &= \frac{1}{(1/2)} \sum_{i=1}^n \P(G_2 \cap G_1 | U_i)\overbrace{\P(U_i)}^{1/n} \\
                    \text{(Chain rule)} 
                        &= \frac{1}{(1/2)} \sum_{i=1}^n \P(G_1|U_i)\P(G_2|G_1 \cap U_i) \cdot \frac{1}{n} \\
                        &= \frac{1}{(1/2)} \sum_{i=1}^n \frac{i-1}{n-1} \cdot \frac{i-2}{n-2} \cdot \frac{1}{n} \\
                        &= \frac{1}{(1/2)} \cdot \frac{1}{3} = \frac{2}{3},
            \end{align*}
            where the summations can be computed using a computer algebra
            system such as Wolfram Alpha.

            The key to the solution was to see that we didn't have any way of
            directing computing probabilities of events involving $G_2$ and $G_1$, but
            once we conditioned on $U_i$ (i.e., once we knew which urn the balls came from),
            we could fully use the information we were given.

            Note that our answer is quite a bit higher than $1/2$ even though
            we've already taken away a green ball from our chosen urn.
            Intuitively, this is because drawing a green ball first tells us
            that we are more likely to have chosen an urn with a higher
            concentration of green balls.
    \end{enumerate}
    }
\end{document}
